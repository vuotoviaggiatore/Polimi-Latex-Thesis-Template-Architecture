% PHD thesis template for POLIMI 中英文版本 (2024) by Siqi MIAO (缪思齐)
% 确保使用XeLaTeX,BibTeX编译。
%chapter文件名不能有空格,有空格无法出bib.bbl
%在该模板的基础上完成:https://github.com/galgalesh/masterproef-template
%第一页封皮需要单独加上。
%适合建筑学论文(不支持代码)。
%该模板非常干净简洁,不建议做任何删减。

\documentclass[12pt,a4paper,openany,italian,chinese,english]{extbook}
\usepackage[a4paper,includeheadfoot,margin=2.50cm]{geometry}
\usepackage{xeCJK}
\setCJKmainfont{SimSun} % 设置中文主字体为宋体
\usepackage{fontspec} %可以有外加字体
\setmainfont{Arial} %设置论文字体



\raggedbottom
\renewcommand{\baselinestretch}{1.2}  % 调整行间距
\usepackage{amsmath}%丰富的数学符号和公式排版功能
\usepackage{amsfonts} 
\usepackage[hyphens]{url} % 允许URL换行,可以在URL中自动添加连字符。
\usepackage{graphicx} % 支持图像插入。
\graphicspath{{Images/}} % 设置图像的默认搜索路径。
\usepackage{pdfpages} % 允许包含整个或部分PDF页面。
\usepackage{pdflscape}
\usepackage{svg} 
\usepackage{enumitem} % 提供更多控制列表(项目符号、编号等)布局的选项。
\usepackage{float} % 提高浮动对象(如图表和图片)的定位控制。
\usepackage{caption} % 提供更多自定义图表标题样式的选项。
\usepackage{subcaption} % 支持子图表和子标题。
\captionsetup[table]{labelfont=bf}%所有caption加粗
\captionsetup[figure]{labelfont=bf}
\usepackage{rotating} % 导入旋转包
%目录
\usepackage[toc,page]{appendix} % 改进附录的管理,包括在目录中的显示。
\usepackage{tocloft} % 提供自定义目录、图表列表和表格列表外观的选项。
\setlength{\cftsecindent}{0pt} % 设置章节在目录中的缩进为0。
\setlength{\cftsubsecindent}{0pt} % 设置小节在目录中的缩进为0。
\setlength{\cftfigindent}{0pt} % 设置图表列表中的图表缩进为0。
\setlength{\cfttabindent}{0pt} % 设置表格列表中的表格缩进为0。

\usepackage{parskip} % 在段落之间添加空间,并取消首行缩进。
\usepackage{lipsum} % 生成随机的Lorem Ipsum文本。
\usepackage{siunitx}%可以在文档中使用\num{}命令了。


\urlstyle{same} % URL使用文档的主字体样式。

\usepackage{color} % 支持颜色。
\definecolor{chaptergrey}{rgb}{0.447, 0.561, 0.647} % 章节数字POLIMI标准色
\usepackage[explicit, pagestyles]{titlesec} % 自定义章节标题和页样式。
\titleformat{\chapter}[display]{\bfseries}{\color{chaptergrey}\fontfamily{pbk}\fontsize{80pt}{100pt}\selectfont\thechapter}{0pt}{\Huge #1}
\titlespacing*{\chapter}{0pt}{-80pt}{30pt} % 自定义章节标题的外观。

\newpagestyle{fancy}{} % 定义一个新的页面样式。
\pagestyle{fancy} % 应用定义的页面样式。
\newpagestyle{numberless}{} % 定义一个无页码的页面样式。

%引用
\PassOptionsToPackage{hyphens}{url} % 向url包传递选项,允许URL自动换行。
\usepackage{hyperref} %交叉引用
%\usepackage{url} % 更好地处理和显示URL。
\usepackage{cleveref}
\usepackage{csquotes}%\blockquote引用
\crefname{figure}{Figure}{Figures}
\crefname{table}{Table}{Tables}
\usepackage[super]{natbib} % 使用数字风格的文献引用。
\setcitestyle{super,open={},close={}} % 设置引用为上角标,且不使用括号
\bibliographystyle{abbrv} % 文献引用样式设置。
\usepackage[nottoc]{tocbibind} % 将目录、图表列表、表格列表等加入到目录中,但不包括目录本身。

\usepackage{tikz} % 提供创建图形(如绘图、图表、图形)的强大工具。

%表格
\usepackage{tabularx}
\usepackage{caption}  % 导入 caption 宏包
\captionsetup[table]{justification=raggedright, singlelinecheck=false} % 左对齐表格标题
\usepackage{multirow} % 为了使用 multirow 命令
%%%%
\usepackage{booktabs} % 提供美观的表格线命令(顶线、底线等)。
\usepackage{threeparttable}
\usepackage{array} % 提供额外的表格选项,比如列格式定义。
\renewcommand{\arraystretch}{0.85} % 将行间距设置为默认的0.85倍
\usepackage{longtable}
\usepackage{threeparttablex}%longtable上可以加注释
\AtBeginEnvironment{table}{\footnotesize}%表格字号
\AtBeginEnvironment{tabular}{\footnotesize}
\AtBeginEnvironment{tabularx}{\footnotesize}
\AtBeginEnvironment{threeparttablex}{\footnotesize}
\AtBeginEnvironment{threeparttable}{\footnotesize}
\AtBeginEnvironment{booktabs}{\footnotesize}
\AtBeginEnvironment{longtable}{\footnotesize}

\usepackage{ragged2e} % 提供更灵活的对齐方式,如两端对齐。
\newcolumntype{L}[1]{>{\raggedright\let\newline\\arraybackslash\hspace{0pt}}m{#1}} % 定义一个新的列类型,用于左对齐文本。
\newcolumntype{C}[1]{>{\centering\let\newline\\arraybackslash\hspace{0pt}}m{#1}} % 定义一个新的列类型,用于居中对齐文本。
\newcolumntype{R}[1]{>{\raggedleft\let\newline\\arraybackslash\hspace{0pt}}m{#1}} % 定义一个新的列类型,用于右对齐文本。



%术语表
%\usepackage[toc,acronym]{glossaries} % 用于创建术语表和缩写表。
\usepackage[nomain,acronym]{glossaries}%只使用\newacronym定义的缩写
%\usepackage[acronym, shortcuts]{glossaries}
\makeglossaries % 初始化术语表处理。
\loadglsentries{FrontBackmatter/definitions.tex} % 引入术语定义文件

%%%%%%%%%%%%%%%%%%%%%%%%%%%%%%%%%%%%%%%%%%%%%%%%%%%%%%%%%%%%%%%%%%%%%%
%
% Add the specific info for your thesis
%
%%%%%%%%%%%%%%%%%%%%%%%%%%%%%%%%%%%%%%%%%%%%%%%%%%%%%%%%%%%%%%%%%%%%%%

\title{	}
\author{Siqi MIAO}




 % 引入个人信息文件,通常包含论文标题、作者等信息。

\begin{document}
%前言
	\frontmatter % 文档前言部分,页码使用罗马数字。
	\pagestyle{empty} % 前言部分页面样式为空(不显示页眉页脚)。

%	\cleardoublepage\include{FrontBackmatter/Publication}
%	\cleardoublepage\include{FrontBackmatter/Acknowledgments}
	\cleardoublepage%*******************************************************
% Abstract
%*******************************************************
%\renewcommand{\abstractname}{Abstract}
\addcontentsline{toc}{chapter}{Abstract}

\pdfbookmark[1]{Abstract}{Abstract}
\begingroup
\let\clearpage\relax
\let\cleardoublepage\relax
\let\cleardoublepage\relax

\chapter*{Abstract}


\endgroup
%	\cleardoublepage\include{FrontBackmatter/Preface}
	
	\tableofcontents\newpage % 生成目录并开始新一页。
	\listoffigures\newpage % 生成插图目录并开始新一页。
	\listoftables\newpage % 生成表格目录并开始新一页。
	\printglossaries\newpage
	\mainmatter % 文档主体部分,页码使用阿拉伯数字。
	\pagestyle{fancy} % 主体部分使用自定义的页眉页脚样式。
	
% 引入更多章节文件。
%	\part{Introduction}
%	\chapter{Introduction}

\section{Statement}
\label{sec:sta}




 
%	\part{Theory and Research Context}
%	\include{chapters/chapter2} 
%	\include{chapters/chapter3} 
%	\include{chapters/chapter4}
%	\part{Assessment}
%	\include{chapters/chapter5}
%	\include{chapters/chapter6}
	\include{chapters/chapter7}
%	\include{chapters/chapter8}
%	\part{Conclusion}
	\include{chapters/chapter9}
	\include{chapters/chapter10}
%%%附录	
	\renewcommand\bibname{Reference}
\bibliography{bib/Bibliography.bib}
 % 参考文献。
	\pagestyle{numberless} % 附录部分使用无页码样式。
	\pagestyle{empty} % 选择一个页面样式。
%	%%********************************************************************
%% Appendix
%%*******************************************************
%% If problems with the headers: get headings in appendix etc. right
%\markboth{\spacedlowsmallcaps{Appendix}}{\spacedlowsmallcaps{Appendix}}
%%************************************************
\chapter{Appendix}
\section{Literature review on public space}


\section{Questionnaire Content}
\includepdf[pages=-]{PDF/questionnaire.pdf}

 % 附录。
%%	\include{chapters/all_chapters_tables}
	
%%	
\end{document} % 文档结束。